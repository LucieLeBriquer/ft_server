\input{pack.tex}

\title{\LARGE \textbf{Analyse}\\ Chapitre 2 : Étude autour des fonctions \\continues sur un compact}
\author{\large Lucie Le Briquer}
\date{\today}

\begin{document}
\maketitle

\dl $\Cl(X,F)$ où $X$ métrique compact et $F$ métrique.

\prp{Rappel}{
  \begin{itemize}
    \item Si $X$ métrique compact, $F$ est métrique.
      On munit $\Cl(X,F)$ de la distance :
      $$d(f,g)=\sup_{x\in X}\ d_F(f(x),g(x))$$
    \item Si $F$ est complet, $d$ rend $\Cl(X,F)$ complet.
    \item Si $F$ est un Banach, alors $\Cl(X,F)$ est un Banach pour $\|f\|_\infty=\underset{x\in X}{\sup}\|f(x)\|_F$
  \end{itemize}
}

\section{Compacts de $\Cl(X,F)$}

\dfn{équicontinuité}{Soit $\A\subset\Cl(X,F)$.\dl 
\begin{itemize}
  \item $\A$ est équicontinue en $x\in X$ si :
    $$\forall\eps>0,\exists\eta>0,\forall y\in X,\ d(x,y)<\eta\Ra\forall f\in\A, d(f(x),f(y))<\eps$$
  \item $\A$ est uniformément équicontinue si :
    $$\forall\eps>0,\exists\eta>0,\forall x,y\in X,\ d(x,y)<\eta\Ra\forall f\in\A, d(f(x),f(y))<\eps$$
\end{itemize}
}

\exot{$E,F$ espaces métriques
\begin{enumerate}
  \item Montrer que si $\A$ finie, alors $\A$ est équicontinue en tout point.
  \item Montrer que si $\A$ ne contient que des fonctions $k-$lipschitziennes, $\A$ est uniformément équicontinue.
  \item Montrer que si $\A$ est équicontinue sur un compact $X$, $\A$ est uniformément équicontinue sur $X$.
\end{enumerate}
}{
\begin{enumerate}
  \item Soit $x\in X$, $\A=\{f_1,...,f_n\}\subset\Cl(X,F)$. Si $\eps>0,\exists\eta_1,...,\eta_n$ tels que :
    $$d(x,y)<\eta_i\Ra d(f_i(x),f_i(y))<\eps$$
    On prend alors $\eta=\min\eta_i$.
  \item Soit $\eps>0$, on pose $\eta=\frac\eps k$. Alors si :
    $$d(x,y)\le\eta,\ d(f(x),f(y))<kd(x,y)\le k\eta\le\eps$$
  \item Par l'absurde. 
    $$\exists\eps>0,\ \forall n>0,\ \exists (x_n,y_n)\in X^2,\quad d(x_n,y_n)<\frac 1 n\text{ et }\exists f_n\in\A,\ d(f_n(x_n),f_n(y_n))>\eps$$
    Par compacité, $y_{\pphi(n)}\xrai x$ alors $x_{\pphi(n)}\xrai x$. Mais $\A$ équicontinue en $x$ donc :
    $$\exists\alpha>0,\ \forall u,v\in X,\ d(x,u)<\alpha\text{ et }d(y,u)<\alpha\quad\Ra\quad d(f(u),f(v))<\eps\ \forall f\in\A$$
    Or $x_{\pphi(n)}\xrai x$ et $y_{\pphi(n)}\xrai y$ donc à partir d'un certain rang $N$ :
    $$d(x,x_{\pphi(N)})<\alpha\text{ et }d(x,y_{\pphi(N)})<\alpha\quad\text{mais}\quad d(f_{\pphi(N)}(x_{\pphi(N)}),f_{\pphi(N)}(y_{\pphi(N)}))>\eps$$
    Absurde. Donc $\A$ uniformément équicontinue.
\end{enumerate}
}

\thm{Ascoli}{Soit $X$ métrique compact et $F$ métrique complet. Soit $\A\subset\Cl(X,F)$. Sont équivalents :
\dl\begin{enumerate}
  \item $\A$ est relativement compact (d'adhérence compacte).
  \item $\A$ est équicontinue en tout point et $\forall x\in X,\A_x=\{f(x),\ f\in\A\}$ est relativement compact.
\end{enumerate}
}
\rmk Sert à :
\begin{itemize}
  \item Montrer la compacité d'un opérateur
  \item Extraire des sous-suites convergentes
\end{itemize}

\prvv
$(2)\Ra(1)$:\dl
Soit $(f_n)_{n\in\N}\in\A^\N$. On va montrer que l'on peut extraire une sous-suite de $(f_n)$ convergente.
\begin{itemize}
  \item Comme $X$ est métrique compact, $X$ est séparable. En effet :
    $$X=\bigcup_{x\in X}\Bc\left(x,\frac 1 n\right)=\bigcup_{i=1}^m\Bc\left(x_i^n,\frac 1 n\right)$$
    (cf. poly)
    \dl Soit $D=(x_n)_{n\in\N}$ dénombrable, dense dans $X$.
    \begin{enumerate}
      \item $x_1$.\quad $\A_{x_1}=\{f(x_1), f\in\A\}$ relativement compact. Donc de $(f_n(x_1))_n$ on peut extraire une sous-suite convergente : $f_{\pphi_1(n)}(x_1)\xrai f(x_1)$.
      \item De même de $(f_{\pphi_1(n)}(x_2))_n$ on extrait une sous-suite convergente : $f_{\pphi_1\circ\pphi_2(n)}(x_2)\xrai f(x_2)$
      \item Pour $x_p$. $(f_{\pphi_1\circ...\circ\pphi_{p-1}(n)}(x_p))_{n\in\N}$ est relativemen compact. On extrait $f_{\pphi_1\circ...\circ\pphi_p(n)}(x_p)\xrai f(x_p)$. On pose $\psi(n)=\pphi_1\circ...\circ\pphi_n(n)$ (procédé d'extraction diagonale).
        \dl On vérifie que : 
          $$\forall p\in\N,\ f_{\psi(n)}(x_p)\xrai f(x_p)$$
    \end{enumerate}
  \item On veut prolonger $f$ sur $X$ grâce au théorème de prolongement vu au TD1. $\ra$ montrons que $f$ est uniformément continue sur $D$.
    \dl Soit $\eps>0$. Comme $\A$ est équicontinue sur $X$ compact, elle est uniformément équicontinue. Donc $\exists\eta>0,\ d(x,y)<\eta\Ra\forall f\in\A\ d(f(x),f(y))<\eps$. Soit $x_k,x_l\in D$ tels que $d(x_k,x_l)<\eta$ :
    \begin{align*}
      d(f(x_k),f(x_l))&\le d(f(x_k),f_{\psi(n)}(x_k))+d(f_{\psi(n)}(x_k),f_{\psi(n)}(x_l))+d(f_{\psi(n)}(x_l),f(x_l))\\
        &\le \eps + \eps +\eps = 3\eps
    \end{align*}
    Donc $f$ est uniformément continue sur $D$. Donc par théorème de prolongement, $f$ se prolonge en une fonction uniformément continue pour tout $X$.
  \item Montrons que $(f_{\psi(n)})$ converge uniformément vers $f$ sur $X$. \\ Soit $\eps>0$, $x\in X$\ $d(f(x),f_{\psi(n)}(x))\le\eps$.
    \dl \emph{Idée.}\quad $d(f(x),f_{\psi(n)}(x))\le d(f(x),f(x_k))+d(f(x_k),f_{\psi(n)}(x_k))+d(f_{\psi(n)}(x_k),f_{\psi(n)}(x))$. 
    \\ Soit $\eta$ associé à l'uniforme équicontinuité de $\A$. On sait que $X=\bigcup_{k\in\N}\Bc(x_k,\eta)$ (où $(x_k)_{k\in\N}$ sous-ensemble dense à partir duquel $f$ est construite). Par compacité, 
    $$X=\bigcup_{i=1}^n\Bc(x_{k_i},\eta)$$
    Soit $i$, $d(x,x_{k_i})<\eta$.
    $$d(f(x),f_{\psi(n)}(x))\le \underset{\le\eps\text{ (UF)}}{\underbrace{d(f(x),f(x_{k_i}))}}+\underset{\le\eps\text{ (APCR idp de }x)}{\underbrace{d(f(x_{k_i}),f_{\psi(n)}(x_{k_i}))}}+\underset{\le\eps}{\underbrace{d(f_{\psi(n)}(x_{k_i}),f_{\psi(n)}(x))}}$$
\end{itemize}

\dd $(1)\Ra(2)$ :
\dl $\A$ relativement compact. En particulier, $\A$ est précompact. Soit $\eps>0$, $\A=\bigcup_{i=1}^n\Bc(f_i,\eps)$.
\\ Soit $\eta_1,...,\eta_n$ associés à l'uniforme continuité des $f_i$. Posons $\eta=\min\eta_i$. 
\\ Soient $x,y\in X$ tels que $d(x,y)<\eta$. Soit $f\in\A$, $\exists i$ tel que $d(f,f_i)<\eps$.
$$d(f(x),f(y))\le \underset{\le\eps}{\underbrace{d(f(x),f_i(x))}}+\underset{\le\eps\text{ (UF de }f_i)}{\underbrace{d(f_i(x),f_i(y))}}+\underset{\le\eps}{\underbrace{d(f_i(y),f(y))}}\le 3\eps$$
Donc $\A$ est uniformément équicontinue. De plus, $\forall x,\ \A_x$ est relativement compact puisque $\A$ est relativement compact. D'où l'équivalence.
\prvf

\rmk Même équivalence sans l'hypothèse $F$ complet.

\dd\textbf{Contre-exemples.}
\begin{itemize}
  \item Si $\A_x$ n'est pas relativement compact. Sur $\Cl([0,1])$, $f_n(x)=n\ \forall x\in[0,1]$.
  \item Si $\A$ n'est pas équicontinue en tout point. $f_n(x)=\sin(nx)$ si $f_{\psi(n)}\xrai f$ uniformément. On a :
    $$\forall\pphi\in\Cl([0,\pi])\quad\int_0^\pi \pphi(x)\sin(nx)dx\xrai 0$$
    Alors $$\forall\pphi\in\Cl([0,\pi])\quad\int_0^\pi\pphi(x)f(x)dx=0$$
    et donc $\int_0^\pi f(x)^2dx=0$
\end{itemize}

\exot{On munit $\Cl^1([a,b])$ de la norme $\|f\|=\|f\|_{+\infty}+\|f'\|_{+\infty}$.
$$i\co (\Cl^1([a,b]),\|.\|)\lra(\Cl([a,b]),\|.\|_\infty)\quad\text{(l'identité)}$$
Montrer que $i$ est une application continue compacte.
}{
  \begin{itemize}
    \item \emph{Continuité.}\quad $\forall f\in\Cl^1([a,b])$, 
      \begin{align*}
        \underset{=}{\|i(f)\|_\infty}&\le\|f\|\\
        \|f\|_\infty &\le\|f\|_\infty+\|f'\|_\infty
      \end{align*}

    \item \emph{Compacité.}\quad Montrons que $i$ est compacte. Soit $B=\overline{\Bc(0,1)}$ pour $\|.\|$ dans $\Cl^1([a,b])$. Montrons que $i(B)=B$ (d'un point de vue ensembliste) est relativement compact dans $(\Cl([a,b]),\|.\|_\infty)$.
      \dl Si $x\in[a,b]$, $\A_x=\{f(x),\ f\in B\}$, $\|f\|_\infty\le\|f\|\le 1$. Donc $\forall x$, $\A_x$ est borné dans $\R$ donc relativement compact.
    \item Montrons que $B$ est équicontinue. Par les inégalités des accroissements finis :
      $$|f(x)-f(y)|\le\|f'\|_\infty|x-y|\quad\text{comme}\quad\|f'\|_\infty\le 1\ \text{car }f\in B$$
  alors $\forall x,y,\ |f(x)-f(y)|\le|x-y|$.
      \\ Donc $B$ est composée de fonctions $1-$lipschitzienne. Donc $B$ est uniformémen équicontinue. Donc par Ascoli, $B$ est relativement compact dans $\Cl([a,b])$.
  \end{itemize}
}
\rmk Si $\dim F<+\infty$, $\A_x$ est relativement compact $\Lra$ $\A_x$ est borné.

\exot{
  Soit $X,Y$ compact métrique de $\R^n$, $K\in\Cl(X\times Y)$. Pour $f\in\Cl(X)$, on définit : 
  $$Tf(y)=\int_X K(x,y)f(x)dx$$
\begin{enumerate}
  \item Montrer que $T$ est un opérateur de $\Cl(X)$ dans $\Cl(Y)$.
  \item Montrer que $T$ est compact.
\end{enumerate}
}{
  \begin{enumerate}
    \item \emph{Définition.}\quad $Tf\in\Cl(Y)$ car $\forall x\in X,\ y\mapsto K(x,y)f(x)\in\Cl(Y)$, et :
      $$\forall y\in Y,\ \forall x\in X,\ |K(x,y)f(x)|\le\|K\|_\infty\|f\|_\infty\quad\text{intégrable sur }X$$
      Donc par théorème de continuité, $Tf\in\Cl(Y)$.
      \dl \emph{Linéarité.}\quad Évident
      \dl \emph{Continuité.}\quad On a :
      $$|Tf(y)|\le\int_X|K(x,y)||f(x)|dx\le\|K\|_\infty\|f\|_\infty\vol(X)$$
      Donc $\|Tf\|_\infty\le\|K\|_\infty\|f\|_\infty\vol(X)$. Donc $T$ est un opérateur.

    \item Soit $B=\overline{\Bc(0,1)}$ dans $(\Cl(X),\|.\|_\infty)$. Montrons que $T(B)$ est relativement compact dans $(\Cl(Y),\|.\|_\infty)$.
      \begin{itemize}
        \item $\A_y=\{Tf(y),\ f\in B\}$
          $$|Tf(y)|\le\int_X|K(x,y)|\underset{\le 1}{\underbrace{|f(x)|}}dx\le\|K\|_\infty\vol(X)$$
          Donc $\A_y$ est borné donc relativement compact.
        \item Soit $\eps>0$, $y\in Y$, et $\eta$ associé à l'uniforme continuité $(x,y)\mapsto K(x,y)$ ($|x-x'|+|y-y'|<\eta\Ra|K(x,y)-K(x',y')|\le\eps$). Soit $y'\in Y$, $|y-y'|<\eta$ alors :
          $$|Tf(y)-Tf(x)|=\left|\int_X (K(x,y)-K(x',y'))f(x)dx\right|\le \eps\int_X\underset{\le 1}{\underbrace{\|f\|_\infty}} dx\le \eps\vol(X)$$
          Donc $(Tf)_{f\in B}$ est équicontinue en $y$. Par Ascoli, $T(B)$ est relativement compact.
      \end{itemize}
  \end{enumerate}
}

\section{Théorème de Stone-Weierstrass}

\thm{Dini}{$X$ espace métrique compact.
$$(f_n)_{n\in\N}\in\Cl(X)^\N\text{ telle que }f_n\xrai f\text{ simplement, }f\text{ continue et }\forall n\in\N,\ f_{n+1}\ge f_n$$
Alors la convergence est uniforme.
}

\prvv
$$\Omega_n=\{x\in X\ |\ f_n(x)>f(x)-\eps\}$$
Par continuité des $f_n$ et de $f$, $\Omega_n$ est ouvert. Par croissance des $(f_n)$, $\Omega_n\subset\Omega_{n+1}$. Par convergence simple, $X=\bigcup_{n\in\N}\Omega_n$. Comme $X$ est compact, $X=\bigcup_{i=1}^m\Omega_{n_i}=\Omega_{n_m}$ (en supposant les $n_i$ croissants).
\dl Donc, $\forall n\ge n_m$, $\forall x\in X$, $f(x)-f_n(x)<\eps$. Et comme $f\ge f_n$ (par croissance des $(f_n)$),
$$\forall n\ge n_m,\ \forall x\in X,\ |f(x)-f_n(x)|<\eps$$
D'où la convergence uniforme.
\prvf

\thm{Stone-Weierstrass}{Soit $X$ métrique compact. $\A\subset\Cl(X)$, $\A$ sous-algèbre de $\Cl(X)$, unitaire et séparante.
$$\text{(séparante)}\quad \forall x,y\in X,\ x\neq y,\ \exists f\in\A\text{ tq } f(x)\neq f(y)$$
Alors $\A$ est dense dans $\Cl(X)$.
}

\lem{}{$\exists (P_n)_{n\in\N}\in\R[X]^\N\ |\ P_n\xrai |\ |$ uniformément sur $[-1,1]$}
\prv (du lemme)\dl
En effet en prenant :
$$\sys{P_0=0\\ P_{n+1}(x)=P_n(x)+\frac 1 2(x^2-P_n(x)^2)\ \forall x\in[-1,1]}$$
On montre que $\forall n\in\N$, $0\le P_n(x)\le P_{n+1}(x)\le|x|$ $\forall x\in[-1,1]$.
\\ Comme $(P_n(x))$ est croissante et majorée, $(P_n(x))$ converge vers $f(x)$ qui vérifie :
$$f(x)=f(x)+\frac{1}{2}(x^2-f(x)^2)$$
donc $f(x)^2=x^2$ et $f(x)\ge 0$. Donc $f(x)=|x|$. Donc $P_n\xrai|\ |$ simplement et $(P_n)$ croissante. Donc par Dini on a la convergence uniforme.
\prvf

\prv (du théorème de Stone-Weierstrass)
\dl On va utiliser les 2 arguments suivants :
\begin{enumerate}
  \item Si $f,g\in\A$, montrons que $\min(f,g)$ et $\max(f,g)$ sont dans $\overline\A$.
    \dl Si $f\in\A,\ f\neq 0$ alors $|f|\in\overline A$. En effet, $\frac{f}{\|f\|_\infty}$ à valeurs dans $[-1,1]$ et $P_n\left(\frac{f}{\|f\|_\infty}\right)\in\A$. Par convergence uniforme, $\left|\frac{f}{\|f\|_\infty}\right|\in\overline\A$. Donc $|f|\in\overline\A$. Or :
    $$\max(f,g)=\frac{f+g}{2}+\frac{|f-g|}{2}\in\overline\A\quad\text{et}\quad\min(f,g)=\frac{f+g}{2}-\frac{|f-g|}{2}\in\overline\A$$
  
  \item Soit $\alpha,\beta\in\R$ avec $\alpha\neq\beta$, et $x,y\in X$. Montrons qu'il existe $u\in\A$ tel que $u(x)=\alpha$ et $u(y)=\beta$.
    \dl En effet, il existe $v\in A$, $v(x)\neq v(y)$ et le système :
    $$\sys{\lambda v(x)+\mu=\alpha\\ \lambda v(y)+\mu=\beta}\quad\text{est de Cramer}$$
    $$\begin{pmatrix} v(x) & 1\\v(y)&1\end{pmatrix}=\begin{pmatrix}\alpha\\\beta\end{pmatrix}$$
    D'où l'existence d'un tel $u$.
\end{enumerate}
Soit $f\in\Cl(X)$, $\eps>0$. Soit $x\in X$. $\forall y\in X$, il existe $u_y\in\A$ tel que $u_y(x)=f(x)$ et $u_y(y)=f(y)$ par $(2)$. On pose :
$$O_y=\{x'\in X\ |\ u_y(x')<f(x')+\eps\}$$
$u_y$, $f\in\Cl(X)$ donc $O_y$ est ouvert et $x,y\in O_y$.
$$X=\bigcup_{y\in X, y\neq x}O_y$$
Or $X$ compact donc il existe $y_1,...,y_n\in X$ tel que $X=\bigcup_{i=1}^n O_{y_i}$. 
\\ On pose $v_x=\min_{1\le i\le n}u_{y_i}\in\overline\A$. Et, $\forall x'\in X$, 
$$v_x(x')=\min_{1\le i\le n}u_{y_i}(x')<f(x')+\eps$$

\dd Posons :
$$\forall x\in X,\quad \Omega_x=\{x'\in X\ |\ v_x(x')>f(x')-\eps\}$$
$\Omega_x$ est un ouvert, $x\in\Omega_x$ donc $X=\bigcup_{x\in X}\Omega_x$. Donc il exists $x_1,...,x_n\in X$ tels que $X=\bigcup_{i=1}^n\Omega_{x_i}$. 
\dd On pose alors $v=\max_{1\le i\le n}v_{x_i}\in\overline\A$.
\dl Alors $\forall x\in X$, $v(x)\ge f(x)-\eps$ et $v(x)<f(x)+\eps$. Donc $|v(x)-f(x)|<\eps\ \forall x\in X$. Donc $v\in\overline\A$ et $\|v-f\|_\infty\le\eps$. Donc $\overline\A=\Cl(X)$.
\prvf

\rmks (conséquences)
\begin{itemize}
  \item Stone-Weierstrass $\Ra$ Weierstrass : les polynômes sont denses dans $\Cl([a,b])$, il suffit de vérifier que l'ensemble est bien une sous-algèbre.
  \item Les fonctions Lipschitziennes sont denses dans $\Cl([a,b])$.
  \item Les polynômes sont-ils denses dans $\Cl(X,\Ci)$ ($X$ compact de $\Ci$) ?
    \\ $X=S^1=\{z\in\Ci,\ |z|=1\}$\quad\quad Soit $f\co z\mapsto \frac 1 z\in\Cl(S^1,\Ci)$
    \dl Si on a $f_n\xrai f$ uniformément sur $S^1$ on aurait :
    $$\int_{S^1}P_n(z)dz=0\xrai\int_{S^1}\frac{dz}{z}=2i\pi\quad\quad\text{absurde}$$
  \item Si $\A$ est stable par conjugaison, le théorème de Stone-Weierstrass reste vrai dans $\Cl(X,\Ci)$ pour $X$ un compact de $\Ci$. (se montre juste avec $\Re(u)=\frac{u+\bar u}{2}$)
  \item Soit $T=\R_{/2\pi\Z}$, les polynômes trigonométriques sont denses dans $\Cl(T,\Ci)$ (puis $T$ vérifie bien la stabilité par conjugaison).
\end{itemize}

%% (Cours du 10 octobre) %%

\exe
$$S=\left\{\sum_{n=-N}^N c_ne^{inx}\ ;\ c_n\in\Ci\right\}$$
$S$ algèbre, unitaire, $T(x)\neq T(y)$ si $x\neq y$ avec $T(x)=e^{ix}$. Alors $S$ est dense dans $\Cl^0(\To,\Ci)$ muni de la norme $\|.\|_\infty$.

\subsection{Transformée de Fourier}
\dl Soit $L^1(\To)$ l'ensemble des fonctions $2\pi-$périodiques intégrables. Soit $f\in L^1(\To)$. Pour $n\in\Z$ on peut définir :
$$\hat f(n)=\int_0^{2\pi} e^{-inx}f(x)dx$$
La suite $(\hat f(n))_{n\in\Z}$ appartient $l^\infty(\Z)$. 
\\ En fait $(\hat f(n))_{n\in\Z}\in c_0(\Z)$, l'espace des suites de limite nulle (lemme de Riemann-Lebesgue). En effet, si $f\in\Cl^1(\To)$ (fonctions $\Cl^1$ $2\pi-$périodiques), alors :
$$\hat f(n)=\int\frac{1}{-in}\partial_X(e^{-inx})f(x)dx\quad\quad\text{puis }O\left(\frac 1 n\right)\text{ par IPP}$$
\dl Notons :
$$\F\co\syst{L^1(\To)&\lra&c_0(\Z)\\ f&\longmapsto&(\hat f(n))_{n\in\Z}}$$

\lem{}{$\F$ est injective.}
\prvv
Si $\F(f)=0$ alors $\int f(x)T(x)dx$ pour tout $T\in S$. Donc $\int f(x)g(x)dx=0$ pour tout $g\in \Cl^0(\To,\Ci)$ par densité. On approche ensuite $\frac{\bar f}{|f|(+\eps)}$ donc $f=0$. 
\prvf
\dd\emph{Est-ce que $\F$ est surjective ?}\quad cf. Chapitre 3

\end{document}
