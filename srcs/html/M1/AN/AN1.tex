\input{pack.tex}

\title{\LARGE \textbf{Analyse}\\ Chapitre 1 : Calcul différentiel et théorèmes de point fixe}
\author{\large Lucie Le Briquer}
\date{\today}

\begin{document}
\maketitle
\dd\tableofcontents

\newpage\section{Rappels de compacité}
\dl Nous allons nous intéresser dans ce cours à la résolution d'équations du type $\phi(x)=0$.

\thm{Riesz}{Soit $E$ un evn. Si la boule unité fermée est compacte, alors $E$ est de dimension finie. 
}

\lem{}{Soit $F$ un sous-espace fermé de $E$, différent de $E$. Alors pour tout réel $r\in]0,1[$, il existe $u\in E$ tel que $\|u\|=1$ et $d(u,F)\ge r$.
}
\prvv
Soit $x\in E\bsl F$. On a $r\in]0,1[$ donc $\frac 1 r d(x,F)> d(x,F)$. Alors $\exists\ y\in F$ tel que $\|x-y\|\le\frac{d(x,F)}{r}$. Posons $u=\frac{1}{\|x-y\|}(x-y)$.
\dl On a bien $\|u\|=1$ et $d(u,F)\ge r$ car pour tout $z\in F$ :
\begin{align*}
  \|u-z\|&=\left\|\frac{1}{\|x-y\|}(x-y)-z\right\|\\
  &=\frac{1}{\|x-y\|}\left\|x-y-\|x-y\|z\right\|\\
  &\ge\frac{1}{\|x-y\|}d(x,F)\quad\text{puisque } y+\|x-y\|z\in F
\end{align*}
Donc $\|u-z\|\ge r$.
\prvf

\prv (du théorème de Riesz)
\dl Par contraposition, supposons $E$ de dimension infinie. On cherche à construire une suite $(u_n)_{n\in\N}$, bornée, sans valeur d'adhérence. On en déduira le théorème. On construit $(u_n)$ telle que :
\begin{enumerate}
  \item $\forall n,\ \|u_n\|=1$
  \item $\forall n,m,\ n\neq m\Ra \|u_n-u_m\|\ge\frac 1 2$
\end{enumerate}
On choisit $u_0$ de norme 1, puis par récurrence on définit $u_{n+1}$ grâce au lemme appliqué à $F_n$ le sev engendré par $u_0,...,u_n$.
\prvf

\section{Schémas itératifs}
\dl Soit $X$ un espace de Banach (evn complet pour la distance induite de sa norme) et $\phi\co X\ra X$. On veut résoudre $\phi(x)=0$.

\subsection{Schéma de Picard}

\dfn{}{Soit $(E,d)$ un espace métrique et $k\in]0,1[$. Une application $F$ est $k$-contractante si
$$\forall (x,y)\in E\times E,\ d(F(x),F(y))\le kd(x,y)$$
}

\thm{Picard}{Soit $(E,d)$ un espace métrique complet et $F$ une application $k$-contractante, $k\in]0,1[$. Alors $F$ admet un unique point fixe.
$$\exists!\ x_{\ast}\in E\ F(x_{\ast})=x_{\ast}$$
}
\prvv
Soit $x_0\in E$. Définissons $(x_n)$ par $x_{n+1}=F(x_n)$. Alors $d(x_{N+1},x_N)\le kd(x_N,x_{N-1})$ donc on a :
$$d(x_{N+1},x_N)\le k^N d(x_1,x_0)$$
et
$$d(x_{N+P},x_N)\le \sum_{k=0}^{p-1} k^{N+k} d(x_1,x_0)\le\frac{k^N}{1-k}d(x_1,x_0)$$
Donc la suite $(x_n)$ est de Cauchy. Sa limite vérifie $F(x_{\ast})=x_{\ast}$.
\prvf

\subsection{Schéma de Newton}
\thm{}{Soit $E$ un espace de Banach et posons :
$$B=\{u\in E,\ \|u\|\le 1\},\ 5B=\{u\in E,\ \|u\|\le 5\}$$
Soit $\phi\co 5B\ra 5B$ une application de classe $C^2$. Supposons que $u_0\in B$ est tel que $\phi(u_0)\in B$ et que :\dl
\begin{enumerate}
  \item La différentielle seconde de $\phi$ est bornée sur $5B$ par $c_1$
  \item $\forall u\in 5F,\ \phi'(u)$ est inversible et $\phi'(u)^{-1}$ est bornée sur $5B$ par $c_2$
\end{enumerate}\dl
Alors si $\eps_0=\|\phi(u_0)\|$ est assez petit, la suite $(u_n)$ définie par :
$$u_{n+1}=u_n-\phi'(u_n)^{-1}\phi(u_n)$$
converge vers une solution de $\phi(u)=0$.
}
%TODO schéma Newton
\rmk Attention, il existe $F\co E\ra E$ continue et non bornée sur la boule unité fermée. Soit $(u_n)$ donnée par la démonstration du théorème de Riesz. Définissons :
$$F(x)=\sum_{n\in\N} n\times\max\left\{0,\frac{1}{10}-\|x-u_n\|\right\}.u_E$$
(avec $u_E$ un vecteur quelconque de $E$)
Pour $x$ fixé au plus un terme est non nul. $F$ est continue et $F(u_n)=\frac{n}{10}$, donc $F$ n'est pas bornée sur la boule unité fermée.

\prvv
On montre :
\begin{itemize}
  \item $P_n$\quad\quad $u_n\in 2B$
  \item $Q_n$\quad\quad $\eps_n=\|\phi(u_n)\|\le\frac{(A\eps_0)^{2^n}}{A}$ où $A=c_1c_2^2$
\end{itemize}
On a $\|\phi'(u_n)^{-1}\phi(u_n)\|\le c_2u_n$ par hypothèse. Donc :
$$\|u_{n+1}\|=\left\|\sum u_{k+1}-u_k\right\|+\|u_0\|\le c_2\sum_{k=0}^n\eps_k+\|u_0\|\le 2\quad\text{si } Q_n$$
Pour démontrer $Q_{n+1}$ on va utiliser :
$$\|\phi(u+v)-\phi(u)-\phi'(u)v\|\le c_1\|v\|^2$$
On applique Taylor à l'ordre 2 à $g(t)=\phi(u+tv)$, on a :
$$\phi(u+v)=\phi(u)+\phi'(u)v+\int_0^1(1-t)d^2\phi(u+tv)(v,v)dt$$
Posons $v_n=-\phi'(u_n)^{-1}\phi(u_n)$, ($u_{n+1}=u_n+v_n$), alors :
\begin{align*}
  \left\|\phi(u_n+v_n)-\phi(u_n)+\phi'(u_n)v_n\right\|&\le c_1\|v_n\|^2\\
  \left\|\phi(u_{n+1})-(\phi(u_n)-\phi(u_n))\right\|&\le c_1\|v_n\|^2\\
  \eps_{n+1}&\le c_1c_2^2\eps_n^2
\end{align*}
Ainsi, 
$$\eps_{n+1}\le A\left(\frac{(A\eps_0)^{2^n}}{A}\right)^2\le \frac{(A\eps_0)^{2^{n+1}}}{A}$$
\prvf

\newpage\section{Théorème d'inversion locale}

\dfn{$\Cl^k-$difféomorphisme}{Soient $B_1,B_2$ deux espaces normés, $U_1\subset B_1$, $U_2\subset B_2$ des ouverts. $F\co U_1\ra U_2$ est un $\Cl^k-$difféomorphisme ($1\le k\le +\infty$) si c'est une fonction $F\in\Cl^k(U_1)$ telle que $F\co U_1\ra U_2$ est une bijection et $F^{-1}$ est $\Cl^k$.
}
\rmk $x\longmapsto x^3$ est $\Cl^\infty$ mais $^3\sqrt{.}$ n'est pas $\Cl^\infty$

\lem{séries de Neumann}{Soit $B$ un espace de Banach et $T\in\Lp(B)$ vérifiant $\|T\|_{\Lp(B)}<1$. Alors $\id-T$ est inversible et :
$$(\id-T)^{-1}=\sum_{n=0}^{+\infty}T^n$$
}

\thm{}{Soit $f\co U\ra B_2$ où $U$ est un ouvert de $B$ et où $B_1$ et $B_2$ sont des espaces de Banach. Si $df(x_0)$ est un isomorphisme de $B_1$ sur $B_2$ et si $f\in\Cl^1(U)$, alors $f$ est un $\Cl^1-$difféomorphisme d'un voisinage de $x_0$ vers un voisinage de $f(x_0)$.
}

\prvv
On peut supposer $B_1=B_2$, $x_0=0$, $f(x_0)=0$ et $df(x_0)=\id$ quitte à travailler avec $\tilde f(x)=(df(x_0))^{-1}(f(x_0+x)-f(x_0))$. Soit $\pphi(x)=f(x)-x$. Alors $d\pphi(0)=0$ et il existe $r>0$ tel que $\|d\pphi(x)\|_{\Lp(B)}\le\frac 1 2\ \forall x\in B(0,r)$ par continuité. 
\dl Posons $W=B\left(0,\frac r 2\right)$, $V=B(0,r)\cap f^{-1}(W)$. Montrons que $f\co V\ra W$ est bijective. L'inégalité des accroissements finis implique que :
$$\|\pphi(x)-\pphi(y)\|\le\frac 1 2\|x-y\|\quad\forall x,y\in B(0,r)$$
Donc : 
\begin{align*}
  \|x-y\|&=\|f(x)-\pphi(x)-f(y)+\pphi(y)\|\\
  &\le \|f(x)-f(y)\|+\|\pphi(x)-\pphi(y)\|\\
  &\le \|f(x)-f(y)\|+\frac 1 2\|x-y\|
\end{align*}
$\Ra \|x-y\|\le 2\|f(x)-f(y)\|\ \forall x,y\in V$.
\dl Donc $f|_V$ est injective. Il reste à montrer la surjectivité. Soit $y\in W=B(0,\frac r 2)$. Montrons que $\exists\ x\in V$ tel que $f(x)=y$. Posons $h(x)=y+x-f(x)$. On cherche alors $x\in V$ tel que $h(x)=x$.
\dd Montrons que $h(\overline{B}_r)\subset\overline{B}_r$ et $h$ est $\frac 1 2 -$contractante sur $\overline{B}_r$. Soit $x\in\overline{B}_r$.
\begin{align*}
  \|h(x)\|&=\|y-\pphi(x)\|\\
  &\le\|y\|+\|\pphi(x)\|\\
  &\le\frac r 2 +\frac r 2 =r
\end{align*}
car $\|y\|\le\frac r 2$ et $\|\pphi(x)\|=\|\pphi(x)-\pphi(0)\|\le\frac 1 2\|x\|\le\frac r 2$.
\dl De plus $h(x_1)-h(x_2)=\pphi(x_2)-\pphi(x_1)$ donc $h$ est $\frac 1 2-$contractante. Alors $\exists!$ point fixe $x\in\overline{B}_r$. On a $x=h(x)\in B_r$ et $f(x)=y\in W$ donc $x\in V$.
\dd Enfin, montrons que $f^{-1}\co X\ra V$ est $\Cl^1$. On a $df(x)=1-d\pphi(x)$ puisque $f(x)=x-\pphi(x)$ et $\|d\pphi(x)\|<\frac 1 2$ sur $B(0,r)$ donc $df(x)$ est inversible pour $x\in B(0,r)$ et $\|(df(x))^{-1}\|_{\Lp(B)}\le 2$. On utilise ensuite le lemme.
\dl Montrons que $f^{-1}$ est différentiable. Soit $y\in W$. Notons $x=f^{-1}(y)$ et posons :
$$L=\left(df(f^{-1}(y))\right)^{-1}=(df(x))^{-1}$$
On veut montrer que :
$$\|f^{-1}(y+z)-f^{-1}(y)-Lz\|=o(\|z\|)$$
Introduisons $h$ tel que $f(x+h)=y+z$. Alors :
\begin{align*}
  \|f^{-1}(y+z)-f^{-1}(y)-Lz\|&=\|x+h-x-Lz\|\\
  &=\|L(L^{-1}h-z)\|\\
  &\le 2\|L^{-1}h-z\|\\
  &\le\|df(x)h-f(x+h)+f(x)\|
\end{align*}
car $\|L\|\le 2$ et $z=f(x+h)-y=f(x+h)-f(x)$.
\dl Or $f(x+h)-f(x)-df(x)h=\|h\|\eps(h)$ avec $\underset{\|h\|\ra 0}{\lim} \eps(h)=0$. On conclut car on a déjà vu que :
$$\|z\|=\|f(x+h)-f(x)\|\ge\frac 1 2\|h\|$$
\prvf

\thm{fonctions implicites}{Soient $B_0,B_1,B_2$ trois espaces de Banach, $U$ un voisinage $(x_0,y_0)$ dans $B_0\times B_1$ et $f\co U\ra B_2$ de classe $\Cl^1$. Supposons qu'il existe une application linéaire continue $A\co B_2\ra B_1$ telle que $f'_y(x_0,y_0)\circ A=\id$. Alors il existe $g\in\Cl^1$ au voisinage de $x_0$ telle que $f(x,g(x))=f(x_0,y_0)$. Si de plus $f'_y(x_0,y_0)$ est bijective, alors $g$ est unique.
}
\prvv
Théorème d'inversion locale pour $F(x,y)=(x,f(x,y))$
\prvf

\newpage\section{Théorème de Cauchy-Lispchitz}

\thm{de Cauchy-Lipschitz}{Soit $n\ge 1$, $f\in\Cl^1(\R\times\R^n,\R^n)$. Pour tout $y_0\in\R^n$ il existe $T>0$ tel que il existe une unique fonction $y\in\Cl^1([-T,T],\R^n)$ vérifiant : 
\begin{equation}\sys{y'(t)=f(t,y(t))\quad\forall t\in[-T,T]\\ y(0)=y_0}\label{cauchy}\end{equation}
}
\prvv
Soit $T>0$. $y$ est solution de \eqref{cauchy} ssi la fonction $z(t)=y(Tt)-y_0$ est solution de :
$$\sys{z(t)=Tf(Tt,z(t)+y_0)\quad\forall t\in[-1,1]\\ z(0)=0}$$
Posons $F(T,z)=z'(\cdot)-Tf(Tt,z(\cdot))$. 
$$F\co\syst{\R\times\Cl^1_{\ast}([-1,1])&\lra&\Cl^0([-1,1])\\ 
(T,z)&\longmapsto&F(T,z)(t)=z'(t)-Tf(Tt,z(t))}$$
$F$ est une application $\Cl^1$ entre espaces de Banach où $\Cl^1_{\ast}=\{z\in\Cl^1, z(0)=0\}$. On a :
$$F'_z(0,0)=\frac{d}{dt}$$
Soit 
$$A\co\syst{\Cl^0([-1,1])&\lra&\Cl^1_{\ast}([-1,1])\\ u&\longmapsto & \int_0^t u(s)ds}$$
Alors $F'_z(0,0)\circ A=\id$ et on peut appliquer le théorème des fonctions implicites. $\exists\ g$ définie sur un voisinage de $0_\R$ telle que $F(T,g(T))=F(0,0)=0$. Alors :
$$z=g(T)\quad\text{est solution de}\quad\sys{z'-Tf(Tt,z)=0\\ z(0)=0}$$
\prvf

%% (Cours du 26 septembre) %%
\thm{point fixe de Brouwer}{Soit $\psi\co B\ra B$ où $B=\{x\in\R^n:\|x\|\le 1\}$ continue. Alors $\psi$ admet un point fixe.
}

\lem{}{Soit $\theta\co B\ra\R^n$ continue telle que $\theta=\id$ sur $S^{n-1}$. Alors $B\subset\theta(B)$}
\prv (du théorème à partir du lemme)
\dl Par l'absurde : supposons $\psi(x)\neq x\ \forall x\in B$. On peut tracer une droite de $\psi(x)$ vers $x$ qui intersecte $S^{n-1}$ en un point noté $\pphi(x)$. $\pphi\co B\ra S^{n-1}$ est continue et $\pphi|_{S^{n-1}}=\id$. On peut voir $\pphi$ comme une fonction continue de $B$ dans $B$ qui vaut l'identité sur $S^{n-1}$. D'après le lemme 1, $B\subset\pphi(B)$. Or $\pphi(B)\subset S^{n-1}$. Contradiction.
\prvf

\lem{}{Soit $f$ une fonction continue à support compact. Soit $\pphi\co \R^n\ra\R^n$ une fonction $\Cl^1$ telle que $\pphi|_{\R^n\bsl B}=\id$. Alors :
$$\int_{\R^n}f(\pphi(x))J(x)dx=\int_{\R^n}f(y)dy$$
où $J=\left|\det\left(\frac{\partial\pphi_j}{\partial x_i}\right)\right|$.
}
\prvv
Introduisons $g(y)=\int_{-\infty}^{y_1}f(s,y_2,...,y_n)ds$. Soit $Q=[-c,c]^n$ un cube tel que $\supp f\subset Q$. $f(\pphi(x))=0$ si $x\notin Q$ et ($c\ge 1$).
\dl Notons que $\int f(\pphi(x))J(x)dx=\int(\partial_{y_1}g)(\pphi(x))J(x)dx$.
$$\frac{\partial g}{\partial y_1}(\pphi(x))\det(D\pphi_1,...,D\pphi_n)=\det\left(D(g(\pphi(x)),D\pphi_2,...,D\pphi_n)\right)$$
(en développant $D(g(\pphi))$). On a :
$$\det(D(g(\pphi)),D\pphi_2,...,D\pphi_n)=M_1\partial_{x_1}(g(\pphi))+...+M_n\partial_{x_n}(g(\pphi))$$
Donc,
$$\int f(\pphi)Jdx=-\int g(\pphi)\underset{0}{\underbrace{\left(\partial_1M_1+....+\partial_nM_n\right)}}+\text{Bord}$$
En effet :
\begin{align*}
  \partial_1M_1+...+\partial_nM_n&=\det(D,D\pphi_2,...,D\pphi_n)\\
  &=\det\begin{pmatrix}\partial_1&\partial_1\pphi_2\\\partial_2&\partial_2\pphi_2\end{pmatrix}\quad\text{si }n=2\\
    &=\partial_1\partial_2\pphi_2-\partial_2\partial_1\pphi_2\quad\text{si }n=2\\
  &=0
\end{align*}
\begin{align*}
  \det(D,D\pphi_2,...,\pphi_n)&=\sum_{\sigma}\eps(\sigma)\prod a_{\sigma(j)j}\\
    &=\sum_\sigma\eps(\sigma)\partial_{\sigma(1)}\left(\prod_{j=2}^n\partial_{\sigma(j)}\pphi(j)\right)\\
    &=\sum_\sigma\eps(\sigma)(\partial_{\sigma(1)}\partial_{\sigma(2)}\pphi_2)...(\partial_{\sigma(n)}\pphi_n)+...+\sum_\sigma\eps(\sigma)(\partial_{\sigma(2)}\pphi_2)...(\partial_{\sigma(1)}\partial_{\sigma(n)}\pphi_n)
\end{align*}
Posons :
    $$A=\sum_\sigma\eps(\sigma)(\partial_{\sigma(1)}\partial_{\sigma(2)}\pphi_2)...(\partial_{\sigma(n)}\pphi_n)$$
Soit $\tau$ une transposition :
    $$A=\sum_\sigma\eps(\sigma\circ\tau)(\partial_{\sigma\circ\tau(1)}\partial_{\sigma\circ\tau(2)}\pphi_2)...(\partial_{\sigma\circ\tau(n)}\pphi_n)=-A$$
Donc $A=0$. Et :
$$\int M_1\partial_1(g(\pphi))+...+M_n\partial_n(g(\pphi))=-\int\left[\partial_1M_1+...\partial_nM_n\right]g(\pphi)+\text{Bord}$$
$\int_Q\partial_1(M_1g\circ\pphi)+...+\partial_n(M_ng\circ\pphi)$, or $\pphi(x)=x$ si $|x|\ge 1$ donc $g(\pphi(x))=xg(x)$ si $|x|\ge 1$. Si $|x|\ge x\ge 1$, $|x_2|=c$, $g(\pphi(x))=0$ :
$$g(y)=\int_{-\infty}^{y_1}f(s,y_2,...,y_n)ds$$
Il reste $\int g(c,x_2,...,x_n)=\int\int f(s,x_2,...,x_n)ds dx_2...dx_n$. 
\prvf

\lem{}{Soit $\pphi\in\Cl^2(\R^n,\R^n)$ avec $\pphi(x)=x$ si $|x|\ge 1$. Alors $B\subset\pphi(B)$.}
\prvv
Par l'absurde. Supposons qu'il existe $y_0\in B, y_0\notin\pphi(B)$. $B=\overline{\Bc(0,1)}$ donc $\pphi(B)$ est compact donc fermé. Donc $\R^n\bsl\pphi(B)$ est ouvert donc $\exists r>0$ tel que $\Bc(y_0,r)\subset\R^n\bsl\pphi(B)$.
\dl On applique alors la relation $\int f(\pphi(x))J(x)dx=\int f(y)dy$ avec $f\in \Cl_0^{+\infty}(\R^n)$, $\supp f\subset\Bc(y_0,r)$. On trouve $\int f(\pphi(x))J(x)dx=0$. Absurde dès que $\int f\neq 0$.
\prvf


\prv (du \rlem{3}, démonstration de \emph{Peter Lax})
\dl On prolonge $\pphi$ sur $\R^n$ par $\pphi(x)=x$ si $|x|\ge1$. On approche $\pphi$ par des fonctions $\Cl^2$, $\pphi_n$, avec $\pphi_n(x)=x$ si $|x|>1+\eps_n$. Soit $y\in B$. Alors $\exists\ x_n\in\overline{\Bc(0,1+\eps_n)}$ tel que $\pphi_n(x_n)=y$. Par compacité $x_n$ a une sous-suite qui converge vers $x\in B$, avec $\pphi(x)=y$.
\prvf

\thm{invariance du domaine}{Soit $U\subset\R^n$ un ouvert et $f\co U\ra\R^n$ continue et injective. Alors $f(U)$ est ouvert.
}

\cor{}{Si $\R^n$ est homéomorphe à $\R^m$ alors $n=m$.}

\rmk 
\begin{itemize} 
  \item Il existe $f\co\R\ra\R^2$ surjective et continue (courbe de Péano).
  \item Il existe $f\co\R^2\ra\R$ injective, par exemple :
    $$\syst{[0,1]^2&\lra&[0,1]\\(0,d_1d_2...;\ 0,d_1'd_2'...)&\longmapsto &0,d_1d_1'd_2d_2'...}$$
  \item Le résultat est faux en dimension infinie :
    $$\tau\co\syst{l^\infty(\N)&\ra&l^\infty(\N)\\(x_0,...,x_n,...)&\longmapsto&(0,x_0,...,x_n,...)}$$
  \item $f\co\tau\mapsto(t,0)$ différence de dimension et l'image n'est pas ouverte.
\end{itemize}

\cor{}{Soit $U\subset\R^n$ ouvert et $f\co U\ra\R^n$ une injection continue. Alors $f$ est un homéomorphisme de $U$ sur $f(U)$.}

\prvv
Soit $V$ un ouvert de $U$. Montrons que $g=f^{-1}$ vérifie $g^{-1}(V)$ est ouvert. Or $g^{-1}(V)=f(V)$, ouvert d'après le théorème. Donc $f^{-1}$ est continue. $f$ est un homéomorphisme.
\prvf

\prv (du \rcor{1})
\dl Supposons $\R^n$ homéomorphe à $\R^n$ et $m<n$. Posons $E_m=\R^m\times\{0\}^{n-m}$ et $p\co\R^m\ra\R^n$ tel que $(x_1,...,x_m)\mapsto(x_1,...,x_m,0,...,0)$. Soit $f\co\R^n\ra\R^m$ homéomorphisme. Alors $F(x)=p(f(x))$ est continue et injective. Donc $F(\R^n)$ est ouvert. Or $F(\R^n)\subset E_m$, absurde.
\prvf

\lem{}{Soit $f\co B\ra\R^n$ continue et injective. Alors $f(0)$ appartient à l'intérieur de $f(B)$ où $B=\overline{\Bc(0,1)}$.}

\prv (du \rthm{8} à partir du \rlem{6})
\dl Soit $y_0\in f(U)$, alors $\exists!\ x_0\in U$ tel que $f(x_0)=y_0$. Posons $F(x)=f(x_0+\eps x)$, $F\co B\ra\R^n$ pour $\eps$ assez petit. D'après le lemme, $y_0\in\Int(F(B))$. Or $F(B)\subset f(U)$ donc $f(U)$ est un voisinage de $y_0$. Donc $f(U)$ est un ouvert.
\prvf

\thm{Tietze}{On dit que $X$ est normal si $\forall F_1,F_2\subset X, F_1\cap F_2=\emptyset$ fermés, $\exists\ U_1,U_2$ ouverts disjoints tels que $F_1\subset U_1$. Soit alors $f\co A\ra\R$ continue sur un fermé de $X$, $\exists\ \tilde f\co X\ra \R$ continue telle que $\tilde f|_A=f$.
}

\prv (du \rlem{6})
\dl Soit $f\co B\ra f(B)$ une bijection continue.  On veut montrer que $f(0)\in\intr{f(B)}$. 
\dl Notons que $f^{-1}$ est continue. En effet si $F$ est un fermé de $B$, alors $(f^{-1})^{-1}(F)=f(F)$ est un compact donc fermé. On utilise alors le théorème de Tietze.
\dl Alors $\exists G\co\R^n\ra\R^n$ continue qui prolonge $f^{-1}\co f(B)\ra B$. On a $G(f(0))=f^{-1}(f(0))=0$. $G$ s'annnule !
\dl Montrons que si $\tilde G\co f(B)\ra B$ est telle que $|\tilde G(y)-G(y)|\le 1$ $\forall y\in f(B)$ alors $\exists\ y_0\in f(B)$ tel que $\tilde G(y_0)=0$.
\dl En effet l'application $h(x)=G(f(x))-\tilde G(f(x))$ vérifie : $h$ est continue et $|h(x)|\le 1\ \forall x\in B$. Donc $h\co B\ra B$ a un point fixe d'après Brouwer. Or $h(x)=x-\tilde G(f(x))$ car $G=f^{-1}$ sur $f(B)$. Donc $h(x)=x\ \Ra\ \tilde G(f(x))=0$.
\dl Supposons que $f(0)\notin\intr{f(B)}$. Par continuité de $G$, $\exists\ \eps>0$ tel que $|G(y)|<\frac{1}{10}$ $\forall y\in\Bc(f(0),2\eps)$. $\exists\ c\in\R^n,\ |c-f(0)|<\eps$ et $c\notin f(B)$. On peut supposer $c=0$.
\dl Alors $|f(0)|<\eps$ et $|G(y)|<\frac{1}{10}$ si $y\in\overline{\Bc(0,\eps)}$. On pose :
$$\Sigma=\Sigma_1\cup\Sigma_2\quad\text{avec }\Sigma_1=\{y\in f(\overline B)\ |\ |y|\ge\eps\}\quad\text{et}\quad\Sigma_2=\partial\Bc(0,\eps)$$
\begin{itemize}
  \item $G$ ne s'annule pas sur $\Sigma_1$ car si $y\in f(B),\ G(y)=f^{-1}(y)$ et $G(f(0))=0$. $G$ bijective, $G$ s'annule en $f(0)$, $f(0)\notin\Sigma_1$ car $|f(0)|<\eps$.
  \item Donc $\exists\delta>0$ tel que $|G(y)|>\delta\ \forall y\in\Sigma_1$
  \item $\exists P\co\R^n\ra\R^n$ polynôme tel que $|P(y)-G(y)|<\delta$. $P$ ne s'annule pas sur $\Sigma_1$.
  \item Quitte à modifier $P$ en ajoutant une constante arbitrairement petite, $P$ ne s'annule pas sur $\sigma_2$.
    Soit $$\tilde G(y)=P\left(\max\left\{\frac{\eps}{|y|},1\right\}y\right)$$
    \begin{enumerate}
      \item $\tilde G$ est continue sur $f(B)$ $(0\notin f(B))$.
      \item $\max\left\{\frac{\eps}{|y|},1\right\}\in\Sigma$ si $y\in f(B)$. Donc $\tilde G(y)\neq 0$.
    \end{enumerate}
    Donc $\tilde G(y)\neq 0\ \forall y\in f(B)$.
  \item Si $y\in f(B)$ avec $|y|\ge\eps$ alors :
    $$|G(y)-\tilde G(y)|=|G(y)-P(y)|\le\delta$$
  \item si $y\in f(B)$ et $|y|<\eps$ alors $|G(y)|<\frac{1}{10}$ car $|G(z)|\le\frac{1}{10}\ \forall z\in\overline{\Bc(0,\eps)}$ et $\max\left\{\frac{\eps}{|y|},1\right\}\in\partial\Bc(0,\eps)$ donc :
    \begin{align*}
      |\tilde G(y)|&\le|P(y)-G(y)|+|G(y)|\\
              &\le\delta+\frac{1}{10}
    \end{align*}
    Donc :
    $$|\tilde G(y)-G(y)|\le|G(y)|+|\tilde G(y)|\le\frac{2}{10}+\delta\le 1$$
  \item Donc $|G-\tilde G|\le 1$, $\tilde G$ continue, n'a pas de zéro. Absurde.
\end{itemize}
\prvf

\end{document}
