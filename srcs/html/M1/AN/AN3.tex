\input{pack.tex}

\title{\LARGE \textbf{Analyse}\\ Chapitre 3 : Théorème de Baire et de Banach}
\author{\large Lucie Le Briquer}
\date{\today}

\begin{document}
\maketitle

\dfn{espace de Baire}{Un espace topologique est un espace de Baire si toute intersection dénombrable d'ouverts denses est dense.
}

\thm{}{Tout espace métrique complet est de Baire.
}
\prvv
Soit $(E,d)$ un espace métrique complet et $(U_n)_{n\in\N}$ une suite d'ouverts denses. Posons :
$$U=\bigcap_{n\in\N} U_n$$
$\overline U=E$ signifie que $U$ rencontre tout ouvert : $\forall V$ ouvert $U\cap V\neq\emptyset$.
\begin{itemize}
  \item $U_0$ est dense donc $U_0\cap V\neq\emptyset$, ainsi $\exists x_0\in U_0\cap V$.
  \\ $U_0$ ouvert et $V$ ouvert, donc $U_0\cap V$ ouvert. Ainsi, $\exists r_0>0$ tel que $\Bc(x_0,r_0)\subset U_0\cap V$ et donc $\exists \rho_0>0$ tel que $\overline{\Bc(x_0,\rho_0)}\subset U_0\cap V$.
  \item $U_1$ dense ainsi $U_1\cap\Bc(x_0,\rho_0)\neq\emptyset$ $\Ra \exists\rho_1>0,\exists x_1\in U_1\cap\Bc(x_0,\rho_0)$ tel que $\overline{\Bc(x_1,\rho_1)}\subset U_1\cap\Bc(x_0,\rho_0)$.
    - ...
\end{itemize}
Par récurrence on définit $(x_n)_{n\in\N}$ et $(\rho_n)_{n\in\N}$ tel que :
$$\overline{\Bc(x_{n+1}),\rho_{n+1}}\subset\Bc(x_n,\rho_n)\cap U_{n+1}$$
On peut de plus supposer que $\rho_n\le 2^{-n}$. Ainsi la suite $(x_n)_{n\in\N}$ est de Cauchy puis que pour $p>n$ on a $x_n,x_p\in\Bc(x_n,\rho_n)$. Elle converge donc vers $x$.
$$\forall N,\forall n\ge N,\ x_n\in\overline{\Bc(x_N,\rho_N)}\quad\text{donc}\quad x\in\overline{\Bc(x_N,\rho_N)}$$
Valable $\forall N\ \Ra\ x\in U_N\ \forall N$ et donc finalement $x\in U\cap V$.
\prvf

\cor{}{Une réunion dénombrable de fermés d'intérieur vide est d'intérieur vide.
}
\prvv
Si $F=\bigcup_p F_p$, alors $F^C=\bigcap_p F_p^C$ est dense.
\prvf
\rmk Cette forme est plus facile à utiliser.

\prp{}{Un espace de Banach (qui n'est pas de dimension finie) n'admet pas de base dénombrable.
}
\prvv
Soit $E$ un espace de Banach. Par l'absurde, supposons que $(e_0,...,e_n,...)$ est une base. On pose $F_p=\Vect(e_0,...,e_p)$. Alors $E=\bigcup_{p\in\N}F_p$.  Si tous les $F_p$ étaient d'intérieur vide, par le corollaire on aurait $E$ d'intérieur vide ce qui est absurde.
\\ Donc $\exists p_0$ tel que $\intr{F_p}\neq\emptyset$ : $\exists a,\exists r>0$ tel que :
$$\Bc(a,r)\subset \intr{F_{p_0}}\subset F_{p_0}$$
Alors $\Bc(0,r)\subset F_{p_0}$ par linéarité. Alors $\lambda\Bc(0,r)\subset F_{p_0}\ \forall\lambda>0$. On obtient finalement $E\subset F_{p_0}$. Donc $E=F_{p_0}$, ainsi $E$ est de dimension finie.
\prvf

\thm{de Banach-Steinhauss}{Soit $E$ un espace de Banach et $F$ un espace vectoriel normé. Soit $(T_\alpha)_{\alpha\in A}$ une famille d'applications linéaires continues, $T_\alpha\Lp(E,F)$, simplement bornées i.e. : 
$$\forall x\in E,\ \sup_{\alpha\in A}\|T_\alpha x\|_F<+\infty$$
Alors,
$$\sup_{\alpha\in A}\|T_\alpha\|_{\Lp(E,F)}<+\infty$$
}
\textbf{Rappel.}
$$\|T\|_{\Lp(E,F)}=\sup_{x\neq 0}\frac{\|T x\|_F}{\|x\|_E}$$
$\Lp(E,F)$ est de Banach si $F$ est de Banach.

\prvv
Soit $p\in\N$ et 
$$F_p=\left\{x\in E\ ;\ \forall\alpha\in A, \|T_\alpha x\|_F\le p\right\}$$
$F_p$ est fermé car :
$$F_p=\bigcap_{\alpha\in A}\underset{\text{continue}}{\underbrace{\pphi_\alpha^{-1}}}(\underset{\text{fermé}}{\underbrace{[0,p]}})$$
où $\pphi_\alpha$ est l'application continue $\pphi_\alpha(x)=\|T_\alpha x\|_F$.
$$(T_\alpha)\text{ simplement borné}\quad\Ra\quad \forall x\in E,\exists p\in\N,\ x\in F_p$$
$$\text{Baire }+\ =\bigcup F_p\quad\Ra\quad\exists p_0\in\N\ |\ \exists a\in E,\ \exists r>0 \text{ tq }\Bc(a,r)\subset F_{p_0}$$
Soit $x\in E$.
\begin{align*}
  \|T_\alpha x\|_F &=\left\|T_\alpha\left(\frac{2\|x\|}{r}\left(\frac{r}{2\|x\|}x+a\right)-\frac{2\|x\|}{r}a\right)\right\|_F\\
  &\le\frac{2\|x\|}{r}\left\|T_\alpha\left(a+\frac{r}{2\|x\|}x\right)\right\|+\frac{2\|x\|}{r}\|T_\alpha a\|
\end{align*}
Or $a\in\Bc(a,r)$, $a+\frac{r}{2\|x\|_E}\in\Bc(a,r)$, donc :
$$\|T_\alpha a\|_F\le p_0,\quad\left\|T_\alpha\left(a+\frac{r}{2\|x\|_E}x\right)\right\|_F\le p_0$$
car $\Bc(a,r)\subset F_{p_0}$. Donc :
$$\|T_\alpha x\|_F\le\frac 4 r p_0\|x\|_E\quad\forall\alpha\in A,\ \forall x\in E$$
Ainsi,
$$\sup_{\alpha\in A}\|T_\alpha\|_{\Lp(E,F)}<+\infty$$
\prvf

\thm{de l'application ouverte}{Soit $E$ et $F$ espaces de Banach et $T\co E\ra F$ linéaire et continue. Si $T$ est bijective alors $T^{-1}$ est continue.
}
\prvv
Montrons que $\exists\delta>0$ tel que :
\begin{align*}
  &\Bc_F(0,\delta)\subset T(\Bc_E(0,1))\\
  \Ra\quad& T^{-1}(\Bc_F(0,\delta))\subset T^{-1}(T(\Bc_E(0,1)))\\
  \Ra\quad& T^{-1}(\Bc_F(0,\delta))\subset \Bc_E(0,1)\\
  \Ra\quad& \|y\|_F<\delta\Ra\|T^{-1}y\|_E<1\\
  \Ra\quad&\|T^{-1}y\|_E\le\frac 2\delta\|y\|_F\quad\quad(\text{en prenant }\tilde y=\frac{\delta}{2\|y\|}y)\\
  \Ra\quad& T^{-1}\in\Lp(F,E)
\end{align*}

\lem{}{Supposons qu'il existe $c>0$ tel que $\Bc_F(0,c)\subset\overline{T(\Bc_E(0,1))}$. Alors :
$$\Bc_F(0,c/2)\subset T(\Bc_E(0,1))$$}
\prv (du lemme)\dl
Soit $y\in \Bc_F(0,c)$. Montrons que $y\in T(\Bc_E(0,2))$. On a $y\in\overline{T(\Bc_E(0,1))}$, donc $\exists x_0\in\Bc_E(0,1)$ tel que $\|y-Tx_0\|_F<\frac c 2$. Donc $2(y-Tx_0)\in\Bc_F(0,c)\subset\overline{T(\Bc_E(0,1))}$.
$$\Ra\exists x_1\in\Bc_E(0,1)\text{ tel que }\|2(y-Tx_0)-Tx_1\|_F<\frac c 2\quad\quad\Ra\left\| y-T\left(x_0+\frac 1 2 x_1\right)\right\|_F<\frac c 4$$
Par récurrence on définit une suite $(x_n)_{n\in\N}$, avec $x_n\in\Bc_E(0,1)$ telle que :
$$\left\|y-T\left(x_0+\frac 1 2 x_1+...+\frac{1}{2^n}x_n\right)\right\|_F<\frac{c}{2^{n+1}}$$
La série $\sum 2^{-n}x_n$ converge normalement donc converge car $E$ est complet. Sa limite $x$ appartient à $\Bc_E(0,2)$. De plus $y=Tx$ par passage à la limite dans l'expression précédente. 
\dl Donc $\Bc_F(0,c)\subset T(\Bc_E(0,2))$.
\prvf

\dd Alors pour montrer qu'il existe $\delta>0$ tel que $\Bc_F(0,\delta)\subset T(\Bc_E(0,1))$, il suffit de montrer que $\exists c>0$ tel que :
$$\Bc_F(0,c)\subset \overline{T(\Bc_E(0,1))}$$
Introduisons $F_p=\overline{T(\Bc_E(0,p))}$ alors $F=\bigcup_{p\in\N} F_p$.
\begin{align*}
  \text{Baire}\quad&\quad\Ra\exists p_0\in\N\text{ tel que }\intr{F_{p_0}}\neq\emptyset\\
  &\quad\Ra\exists a\in F,\ \exists r>0,\ \Bc(a,r)\subset F_{p_0}
\end{align*}
$$\Bc(a,r)\subset\overline{T(\Bc_E(0,p_0))}=p_0\overline{T(\Bc_E(0,1))}$$
En particulier, $a\in\overline{T(\Bc_E(0,p_0))}$. Donc $\exists x_n\in \Bc_E(0,p_0)$ tels que $a=\lim Tx_n$, alors $-a=\lim T(-x_n)\ \Ra\ -a\in\overline{T(\Bc_E(0,p_0))}$.
\begin{align*}
  &\Ra\ \Bc(a,r)-a\subset\overline{T(\Bc_E(0,2p_0))}\\
  &\Ra\ \Bc(0,r)\subset\overline{T(\Bc_E(0,2p_0))}\\
  &\Ra\ \Bc\left(0,\frac{r}{2p_0}\right)\subset \overline{T(\Bc_E(0,1))}
\end{align*}
\prvf

\cor{}{Soit $E,F$ des espaces de Banach et $(T_n)_{n\in\N}$ une suite de $\Lp(E,F)$. Si $(T_nx)_{n\in\N}$ converge pour tout $x$ vers $T(x)$, alors $T\in\Lp(E,F)$.
}
\prvv
$x\mapsto T(x)$ linéaire ok. De plus, $\forall x\in E,\ \sup_{n\in\N}\|T_nx\|_F<+\infty$. Donc :
$$\sup_\N\|T_n\|_{\Lp(E,F)}<+\infty$$
Donc $\exists c>0,\ \forall n\in\N,\ \forall x\in E,\quad \|T_n x\|_F\le c\|x\|_E$
En passage à la limite :
$$\|Tx\|_F\le c\|x\|_E\quad\forall x\in E$$
Donc $T\in\Lp(E,F)$.
\prvf

\cexe Soit $T\co f\mapsto f'$ de $E\ra F$ avec $E=\Big(\Cl^1([0,1]),\|f\|_\infty=\sup_{[0,1]}|f(t)\Big)$ et $F=\Big(\Cl^0([0,1]),\|.\|_\infty\Big)$. $T$ est linéaire mais pas continue. Prendre $\pphi_n=\frac{1}{\sqrt n}\sin(nx)\xrai 0$ dans $E$ mais $\|T\pphi_n\|_F\xrai +\infty$.

\cor{équivalence des normes}{Soit $E$ un espace vectoriel muni de 2 normes $\|.\|_1$ et $\|.\|_2$ de Banach pour ces 2 normes. Si 
$$\exists c>0,\ \forall x\in E,\ \|x\|_1\le c\|x\|_2\quad(\ast)$$
alors :
$$\exists c'>0\ |\ \forall x\in E,\ \|x\|_2\le c'\|x\|_1\quad(\ast\ast)$$
}
\prvv
Considérons :
$$T\co\syst{(E,\|.\|_2)&\lra &(E,\|.\|_1)\\ x&\longmapsto& x}$$
$(\ast)\Ra\ T$ continue. Théorème de l'application ouverte $\Ra\ T^{-1}$ continue $\Ra\ (\ast\ast)$.
\prvf

\exe Soit 
$$T\co\syst{L^1(\To)&\lra&c_0(\Z)\\f&\longmapsto&(\hat f(n))_{n\in\Z}}$$
On a vu $T$  linéaire continue, injective. Supposons $T$ surjective. Alors $T^{-1}$ serait continue :
$$\Ra\quad\exists c>0\ |\ \|T^{-1}u\|_{L^1}\le c\|u\|_{l^\infty}$$
Considérons $u_N=T(D_N)$ où $D_N=\sum_{k=-N}^N e^{ikx}$. On a :
$$u_N=(...,0,...,0,\underset{\text{de }-N\text{ à }N}{\underbrace{1,..,1}},0,...,0,...)$$
car $(u_N)_p=\frac{1}{2\pi}\int_0^{2\pi}e^{-ipx}D_N(x)dx$. Donc $\|u_N\|_{l^\infty}=1$.
\dl Or $T^{-1}(u_N)=D_N$ et $\|D_N\|_{L^1}$
  $$D_N(x)=\sum_{k=-N}^N e^{ikx}=\frac{\sin\left(\left(N+\frac 1 2\right)x\right)}{\sin\left(\frac x 2\right)}$$
  Alors :
\begin{align*}
  \int_0^{2\pi}|D_N(x)|dx&\ge\int_0^{2\pi}\frac{\left|\sin\left(\left(N+\frac 1 2\right)x\right)\right|}{\frac x 2}\\
  &\ge 2\int_0^{(2N+1)\pi}\frac{|\sin x|}{x}dx\xraii{N}+\infty
\end{align*}
Donc $T$ n'est pas surjective.

\thm{du graphe fermé}{Soit $E$ et $F$ deux espaces de Banach et $T\co E\lra F$ linéaire. Alors 
$$T\text{ est continue}\quad\Lra\quad\Gc(T)=\{(x,Tx)\ :\ x\in E\}\quad\text{est fermé dans }E\times F$$
}
\prvv
\begin{enumerate}
  \item $T$ continue $\Ra\Gc(T)$ fermé.
    \\ Soit $(x_n,t_n)\in\Gc(T)$, convergeant vers $(x,y)$ Alors $y_n=T x_n$. $x_n\xrai x$ et $T$ continue $\Ra\ Tx_n\xrai Tx$. Or $y_n=Tx_n$ converge vers $y$. Alors par unicité de la limite on a $y=Tx$, i.e. $(x,y)\in\Gc(T)$. Donc $\Gc(T)$ est fermé.
  \item $\Gc(T)$ fermé $\Ra\ T$ continue.
    \dl Introduisons $N(x)=\|x\|_E+\|Tx\|_F$ (appelée \emph{norme du graphe}).
    $$N(x+y)\le N(x)+N(y)\quad\quad N(\lambda x)=|\lambda N(x)|\quad\quad N(x)=0\Ra x=0$$
    Montrons que $(E,N(.))$ est de Banach. Soit $(x_n)_{n\in\N}$ de Cauchy. Alors $(x_n)$ est de Cauchy dans $(E,\|.\|_E)$. Or $(E,\|.\|_E)$ est complet donc $\exists x\in E$, $\|x_n-x\|_E\xrai 0$. De même $\exists y\in F$ tel que $\|T x_n-y\|_F\xrai 0$. Or $\Gc(T)$ est fermé dans $(E\times F)$ donc $y=Tx$. Donc :
    $$N(x_n-x)=\|x_n-x\|_E+\|Tx_n-y\|_F\xrai 0$$
  Donc $(E,N(.))$ de Banach.
    \dl De plus, on a que $\|x\|_E\le N(x)\ \forall x\in E$. Le corollaire sur l'équivalence des normes implique qu'il existe $c>0$ tel que :
    $$N(x)\le C\|x\|_E\quad\forall x\in E$$
    \begin{align*}
      \Ra\quad &\|x\|_E+\|Tx\|F\le c\|x\|_E\\
      \Ra\quad &\|Tx\|_F\le(c-1)\|x\|_E\\
      \Ra\quad &T\in\Lp(E,F)
    \end{align*}
\end{enumerate}
\prvf

\end{document}
